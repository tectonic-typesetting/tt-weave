% from webmac

\def\mc{}
\def\AT!{@}
\def\30{} % for named module 1009
\def\XeTeX{XƎ\TeX}

\def\ttwDocumentTitle{XeTeX: A Pseudoprogram}

% Custom preamble material:

\tduxSetupOutput{tdux-standalone.html}{module0.html}
\write\modulehtmllisting{<a href="module0.html"></a>}

\textbf{Preamble.} This digital book describes the \XeTeX\ typesetting program,
which converts input \TeX\ source code to the XDV binary format.

\mysection{About XeTeX} % \XeTeX canvas currently doesn't work here

\XeTeX\ is a modified version of Donald E.\ Knuth’s \TeX\ typesetting program
that adds support for Unicode text and OpenType fonts, among other changes. It
is the foundation of the \href{https://tectonic-typesetting.github.io/}{Tectonic
project}, a modernized \TeX\ system, which is built around a modified clone of
the \XeTeX\ engine.

\mysection{About this book}

The original \TeX\ system is written in WEB, a literate programming language —
one that combines code and documentation. WEB code can be turned into either an
executable program (a process called “tangling”) or into a book combining the
formatted source code and extensive documentation (“weaving”). The \TeX\ program
— viewed holistically as both code and documentation — is the work of
Donald E.\ Knuth, and is renowned for its staggering attention to detail,
comprehensiveness, and polish. The associated book is published in hardcopy as
\href{https://www.worldcat.org/title/876762639}{Computers \& Typesetting, Volume
B — \TeX: The Program}.

\textit{This} book is “woven” from the WEB code underlying \XeTeX, which is is
expressed as a series of patches to Knuth’s \TeX\ file. It is the work of many
authors building atop Knuth’s foundation. Unlike \TeX\ it is not the product of
a singular vision, and even the best of us struggle to match the standard of
exactitude set by Knuth. Abrupt changes in style will be apparent if you read
this book linearly. While much of the book's text will say that it is describing
\TeX, it is always and only describing \XeTeX.

However, the full \XeTeX\ program consists of not only WEB code, but also
extensions written in C and C++. Those extensions are not included in this book.
Some of the WEB code can only be understood in the context of its interaction
with the absent extension code. Furthermore, \XeTeX\ itself must be combined
with other programs and data files to do anything useful with a modern \TeX\
document. Those ecosystem-level aspects are not addressed here either.

Furthermore, this book is derived from the WEB code through a program called
\href{https://github.com/tectonic-typesetting/tt-weave/}{tt-weave}, which
transforms the WEB file into \TeX\ code that is then compiled into a
\href{https://vuejs.org}{Vue.js} web application using
\href{https://tectonic-typesetting.github.io/}{Tectonic} and
\href{https://parceljs.org/}{Parcel}. The tt-weave program emits the code
component of the WEB file after reformatting it into a new syntax resembling the
C and Rust languages. Between this rewriting, the absent extension code, and the
fact that the original typesetting targets the printed page, not the screen,
this digital book should be taken as informative \textit{but not definitive}
regarding the workings of \XeTeX. That is why this book's title refers to
\textit{a pseudoprogram}, not \textit{the program}.

The specific provenance of the WEB code used to create this book is the \XeTeX\
source tree maintained by the \href{https://www.tug.org/texlive/}{\TeX\ Live}
project. Files from that tree are processed using scripts from the
\href{https://github.com/tectonic-typesetting/tectonic-staging/}{\textsf{tectonic-staging}}
code repository, with a minimal patch applied in order to create a WEB document
that can be woven. No other modifications are made.

\mysection{About the code in this book}

The WEB language is essentially a sophisticated preprocessor for Pascal.
“Tangling” a WEB program produces a Pascal file. WEB is described in the
Stanford Computer Science report CS-TR-83-980,
\href{http://i.stanford.edu/TR/CS-TR-83-980.html}{The WEB system of structured
documentation}, by Donald E.\ Knuth.

WEB programs consist of a numbered sequence of modules. Each module may contain
descriptive documentation, macro definitions, and/or Pascal code. The tangled
output consists of the concatenation of the Pascal code pieces, with two layers
of processing applied. First, macro definitions are expanded in the code text as
they appear. Second, some modules are “named modules” whose entire text can be
included within other modules. The same name can be defined in multiple modules,
in which case the text of the named module is the concatenation of the text of
all of its constituents. Both layers of processing are applied at the textual
level, and constructs such as imbalanced parentheses occur frequently.

The code as rendered in this book has been parsed and transformed from a
Pascal-like syntax to a pseudocode syntax resembling C and Rust. This rewriting
is superficial and does not attempt to alter aspects of the Pascal language that
appear quirky by present standards, such as the lack of a built-in
\texttt{return} statement. Rust macro syntax (\texttt{macro!}) is used to
indicate forms with non-intuitive meaning. For instance, \texttt{ord!("s")}
indicates the integer ASCII code of the character \texttt{s}, a construct used
instead of classic WEB's distinction between double- and single-quoted strings.

\mysection{Credits}

The original \TeX\ program is by Donald E.\ Knuth. \TeX\ is a trademark of the
American Mathematical Society.

\XeTeX\ was developed by SIL International, Jonathan Kew, Han The Thanh, and
Khaled Hosny.

\XeTeX\ builds upon \eTeX, developed by P.\ Breitenlohner. \eTeX\ is a trademark
of the NTS Group. It also includes the \TeX\ extensions TeX–XeT and Sync\TeX.
The code further includes the changes introduced in ML\TeX, by Michael J.\
Ferguson and B.\ Raichle, but they are not activated in \XeTeX.

% Here, our \href is dumb and needs handholding:
\catcode`~=11
tt-weave and its templates were originally developed by
\href{https://newton.cx/~peter/}{Peter K.\ G.\ Williams}. Much of the HTML and
CSS was inspired by the templates used by the
\href{https://rust-lang.github.io/mdBook/}{mdBook} tool.
\catcode`~=\active

\tduxEmit

% Put these last since they mess up VS Code's syntax highlighting

\def\X#1:#2\X{%
  \ifmmode\gdef\XX{\null$\null}\else\gdef\XX{}\fi % section name
  \XX$\langle\,$#2 #1$\,\rangle$\XX}

\def\[{[}
